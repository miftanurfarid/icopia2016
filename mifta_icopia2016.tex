\documentclass[a4paper]{jpconf}
\usepackage{graphicx}
\begin{document}
\title{Speech Segregation Based-on Binaural Cue: Interaural Time Difference (ITD) and Interaural Level Difference (ILD)}

\author{Mifta Nur Farid, Dhany Arifianto}

\address{Dept. of Engineering Physics, Faculty of Industrial Technology, Institut Teknologi Sepuluh Nopember, Kampus ITS Sukolilo, Surabaya 60111, Indonesia}

\ead{miftanurfarid@gmail.com}

\begin{abstract}
In a conversation at a cocktail party, a person can focus on single conversation even though the background sound and other people conversation is quite loud. This phenomenon is known as the cocktail party effect. In an early study, explained that binaural hearing have an important contribution to the cocktail party effect. So in this study, will be performed separation on the input binaural sound with 2 microphone sensors of two sound sources based on both the binaural cue, interaural time difference (ITD) and interaural level difference (ILD) using binary mask. To estimate value of ITD, is used cross-correlation method which the value of ITD represented as time delay of peak shifting at timefrequency unit. Binary mask is e stimated based on pattern of ITD and ILD to relative strength of target that computed statistically using probability density estimation. Results of sound source separation performing well with the value of speech intelligibility using the percent correct word by 86\% and 3 dB by SNR.
\end{abstract}


\section{Introduction}
The auditory system can seperate many sound source simultaneously. For example, in a conversation at a cocktail party, a person can focus on single conversation even though the background sound and other people conversation is quite loud. This phenomenon is known as the cocktail party effect.
% no \IEEEPARstart

% You must have at least 2 lines in the paragraph with the drop letter
% (should never be an issue)




%\hfill mds
 
%\hfill August 26, 2015

%\subsection{Subsection Heading Here}
%Subsection text here.


%\subsubsection{Subsubsection Heading Here}
%Subsubsection text here.


% An example of a floating figure using the graphicx package.
% Note that \label must occur AFTER (or within) \caption.
% For figures, \caption should occur after the \includegraphics.
% Note that IEEEtran v1.7 and later has special internal code that
% is designed to preserve the operation of \label within \caption
% even when the captionsoff option is in effect. However, because
% of issues like this, it may be the safest practice to put all your
% \label just after \caption rather than within \caption{}.
%
% Reminder: the "draftcls" or "draftclsnofoot", not "draft", class
% option should be used if it is desired that the figures are to be
% displayed while in draft mode.
%
%\begin{figure}[!t]
%\centering
%\includegraphics[width=2.5in]{myfigure}
% where an .eps filename suffix will be assumed under latex, 
% and a .pdf suffix will be assumed for pdflatex; or what has been declared
% via \DeclareGraphicsExtensions.
%\caption{Simulation results for the network.}
%\label{fig_sim}
%\end{figure}

% Note that the IEEE typically puts floats only at the top, even when this
% results in a large percentage of a column being occupied by floats.


% An example of a double column floating figure using two subfigures.
% (The subfig.sty package must be loaded for this to work.)
% The subfigure \label commands are set within each subfloat command,
% and the \label for the overall figure must come after \caption.
% \hfil is used as a separator to get equal spacing.
% Watch out that the combined width of all the subfigures on a 
% line do not exceed the text width or a line break will occur.
%
%\begin{figure*}[!t]
%\centering
%\subfloat[Case I]{\includegraphics[width=2.5in]{box}%
%\label{fig_first_case}}
%\hfil
%\subfloat[Case II]{\includegraphics[width=2.5in]{box}%
%\label{fig_second_case}}
%\caption{Simulation results for the network.}
%\label{fig_sim}
%\end{figure*}
%
% Note that often IEEE papers with subfigures do not employ subfigure
% captions (using the optional argument to \subfloat[]), but instead will
% reference/describe all of them (a), (b), etc., within the main caption.
% Be aware that for subfig.sty to generate the (a), (b), etc., subfigure
% labels, the optional argument to \subfloat must be present. If a
% subcaption is not desired, just leave its contents blank,
% e.g., \subfloat[].


% An example of a floating table. Note that, for IEEE style tables, the
% \caption command should come BEFORE the table and, given that table
% captions serve much like titles, are usually capitalized except for words
% such as a, an, and, as, at, but, by, for, in, nor, of, on, or, the, to
% and up, which are usually not capitalized unless they are the first or
% last word of the caption. Table text will default to \footnotesize as
% the IEEE normally uses this smaller font for tables.
% The \label must come after \caption as always.
%
%\begin{table}[!t]
%% increase table row spacing, adjust to taste
%\renewcommand{\arraystretch}{1.3}
% if using array.sty, it might be a good idea to tweak the value of
% \extrarowheight as needed to properly center the text within the cells
%\caption{An Example of a Table}
%\label{table_example}
%\centering
%% Some packages, such as MDW tools, offer better commands for making tables
%% than the plain LaTeX2e tabular which is used here.
%\begin{tabular}{|c||c|}
%\hline
%One & Two\\
%\hline
%Three & Four\\
%\hline
%\end{tabular}
%\end{table}


% Note that the IEEE does not put floats in the very first column
% - or typically anywhere on the first page for that matter. Also,
% in-text middle ("here") positioning is typically not used, but it
% is allowed and encouraged for Computer Society conferences (but
% not Computer Society journals). Most IEEE journals/conferences use
% top floats exclusively. 
% Note that, LaTeX2e, unlike IEEE journals/conferences, places
% footnotes above bottom floats. This can be corrected via the
% \fnbelowfloat command of the stfloats package.



\section{Data Training Process}
\subsection{Binaural Hearing}

\subsection{Auditory Periphery}

\subsection{Binaural Cue Extraction}

\subsection{Relative Strength Calculation}

\subsection{Probability Density Estimation}

\section{Segregation Process}
\subsection{Azimuth Estimation}

\subsection{Binary Mask Estimation}

\section{Evaluation}
\subsection{Speech Intelligibility}

\subsection{Perceptual Evaluation of Speech Quality (PESQ)}

\section{Result and Discussion}
\section{Conclusion}
The conclusion goes here.


\section*{References}

\begin{thebibliography}{9}
\end{thebibliography}

\end{document}


